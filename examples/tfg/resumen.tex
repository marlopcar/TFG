\chapter{Resumen}

Las mejoras de la medicina en las últimas décadas son muchísimas y la cantidad de enfermedades a las que, poco a poco, se les ha ido ganando terreno, se incrementa casi a diario.

El cáncer es una de esas enfermedades contra las que se está luchando y cada día supone un avance en su tratamiento. La Doctora Blanca Herrero, especialista en oncología, comenzó a registrar todos los datos relacionados con la enfermedad de sus pacientes, que habían sobrevivido a un cáncer infantil, para así, poder realizar un seguimiento de su avance y  estudio con el fin de poder detectar sus necesidades y problemas de forma precoz.

El registro de estos datos, que se hace manualmente, supone una tarea lenta y tediosa. La Fundación Ramón Areces, en colaboración con la Fundación Oncohematología Infantil, deciden regalar la aplicación ONCOSUP al Hospital Infantil Universitario Niño Jesús y permitir así a la Dra. Herrero registrar los datos de los pacientes al mismo tiempo que pasan la consulta y exportarlos después fácilmente, agilizando su trabajo enormemente.

Este trabajo documenta la unión de la autora a un equipo de trabajo en Informática El Corte Inglés a través del convenio FORTE, con el fin de desarrollar la aplicación ONCOSUP, usando tecnologías punteras para tratar de desarrollar una aplicación web completa en un tiempo récord.

El resultado de este trabajo supone un cambio en el estudio oncológico del cáncer infantil, y sienta las bases para la futura adición de funcionalidades de ONCOSUP o su aplicación a otros hospitales, permitiendo así, que estudios similares puedan comenzar.


\chapter{Abstract}

Improvements in medicine in last decades are many and the amount of deseases which, step by step, we have been gaining ground, increases almost daily.

Cancer is one of those deseases that are still being fought and each year is an advance in its treatment. Doctor Blanca Herrero, specialist in oncology, started to register all data related to the desease of her patients, who had survived childhood cancer, in order to be able to track the progress and study their needs as early as possible.

The registration of these patients is a slow and tedious task Ramón Areces Foundation in colaboration with Child Oncohematology Fundation, have decided to give the ONCOSUP application to Hospital Infantil Universitario Niño Jesús as a gift and allow Dr. Herrero to register her patients data at the same time she pass consultation and export them easily, making her work a faster task.

This work documents the union of the author to a work team at Infomática El Corte Inglés via FORTE agreement, in order to develop the functionalities of ONCOSUP, using new technologies to develop an application in a record time.

The result of this work ir a change in the oncological study of childhood cancer, and lays the foundations for the future addition of functionalities or its extension to other hospitals, allowing similar studies to begin.
