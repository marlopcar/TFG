\chapter{Introducción}

\drop{U}{n} paciente de cáncer es considerado superviviente a largo plazo cuando han pasado cinco años desde su diagnóstico y tratamiento. Los distintos avances en medicina han generado un aumento en los supervivientes de cáncer.

Es habitual que los supervivientes a largo plazo desarrollen algún tipo de afección tardía relacionada con el tratamiento oncológico, que puede, en ocasiones, requerir tratamiento. De esta forma, el servicio de Oncología Pediátrica del Hospital Infantil Universitario Niño Jesús, en colaboración con la Fundación Ramón Areces y la Fundación Oncohematología Infantil firman un acuerdo de colaboración para la creación del primer registro de supervivientes a largo plazo de cáncer infantil en España \cite{noticiaRegistro}. El registro de pacientes supervivientes surge con el objetivo de poder realizar un seguimiento de los pacientes, con el fin de poder tener en cuenta su enfermedad oncológica y las posibles complicaciones.

La encargada de la construcción de este registro es la Doctora Blanca Herrero, que tras un tiempo trabajando en él, detectó la necesidad de agilizar el trabajo disponiendo de un sistema informático específico que permitiera la adquisición y procesamiento de los datos.

Hasta el momento, la recopilación de la información se llevaba a cabo manualmente, en papeles manuscritos, de los que posteriormente se pasaba la información a una hoja de cálculo. Todo este trabajo resulta lento y ralentiza el registro de los datos y, en consecuencia, retrasa la obtención de datos en beneficio de los pacientes.

De esta forma se plantea a Informática El Corte Inglés, en adelante IECISA, desarrollar un sistema informático con los requisitos deseados por el servicio de Oncología Pediátrica. Este trabajo es el resultado de la participación de la autora en el desarrollo de este proyecto a través del convenio FORTE \cite{forte}, que permite a los alumnos colaborar en proyectos reales de empresas y tener así un primer contacto con el mundo profesional.

Así pues, en este documento se describe ONCOSUP, una aplicación web para el manejo de datos de pacientes supervivientes a largo plazo de cáncer infantil, que permitirá  entregar al paciente un informe con toda la información referente a su seguimiento y recomendaciones para su tratamiento, además de permitir la exportación sencilla de datos para su posterior estudio con una herramienta de análisis estadístico.
\section{Estructura del documento}

El presente documento sigue la estructura expuesta a continuación:

\begin{definitionlist}
\item[Capítulo \ref{chap:objetivos}: \nameref{chap:objetivos}] Se definen los objetivos del proyecto.
\item[Capítulo \ref{chap:antecedentes}: \nameref{chap:antecedentes}] Se explica el contexto en el que surge la necesidad de ONCOSUP y se explican las diferentes herramientas que se usan en su desarrollo.
\item[Capítulo \ref{chap:metodo}: \nameref{chap:metodo}] Desarrolla el método de trabajo seguido durante el desarrollo del proyecto.
\item[Capítulo \ref{chap:resultados}: \nameref{chap:resultados}] Expone todo el desarrollo del proyecto, haciendo hincapié en el trabajo realizado por la autora.
\item[Capítulo \ref{chap:conclusiones}: \nameref{chap:conclusiones}] Reflexión de la autora de lo acontecido durante el desarrollo del proyecto, tanto del trabajo en IECISA, como de la experiencia del FORTE.
\end{definitionlist}


% Local Variables:
%  coding: utf-8
%  mode: latex
%  mode: flyspell
%  ispell-local-dictionary: "castellano8"
% End:
